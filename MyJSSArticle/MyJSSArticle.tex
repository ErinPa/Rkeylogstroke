\documentclass[article]{jss}
\usepackage[utf8]{inputenc}

\providecommand{\tightlist}{%
  \setlength{\itemsep}{0pt}\setlength{\parskip}{0pt}}

\author{
Nicolas Ballier\\Paris Diderot \And Erin Pacquetet\\Paris Diderot
}
\title{A Capitalized Title: Something about a Package \pkg{Rkeylog}}

\Plainauthor{Nicolas Ballier, Erin Pacquetet}
\Plaintitle{A Capitalized Title: Something about a Package foo}
\Shorttitle{\pkg{Rkeylog}: A Preliminary Exploration}

\Abstract{
The abstract of the article.
}

\Keywords{keywords, not capitalized, \proglang{Java}}
\Plainkeywords{keylogs, bursts, clusters}

%% publication information
%% \Volume{50}
%% \Issue{9}
%% \Month{June}
%% \Year{2012}
%% \Submitdate{}
%% \Acceptdate{2012-06-04}

\Address{
    Nicolas Ballier\\
  Paris Diderot\\
  Université Paris Diderot Paris 7 UFR études anglophones Bât Olympe de
  Gouges Case 7046 5 rue Thomas Mann 75205 Paris Cedex 13First line\\
  E-mail: \email{nicolas.ballier@univ-paris-diderot.fr}\\
  URL: \url{http://www.clillac-arp.univ-paris-diderot.fr/user/nicolas_ballier}\\~\\
    }

% Pandoc header

\usepackage{amsmath}

\begin{document}

\section{Introduction}\label{introduction}

This prorotype describes the information required for the package.

\subsection{Code formatting}\label{code-formatting}

Don't use markdown, instead use the more precise latex commands:

\begin{itemize}
\tightlist
\item
  \proglang{Java}
\item
  \pkg{plyr}
\item
  \code{print("abc")}
\end{itemize}

\section{R code}\label{r-code}

Can be inserted in regular R markdown blocks.

\begin{CodeChunk}

\begin{CodeInput}
R> x <- 1:10
R> x
\end{CodeInput}

\begin{CodeOutput}
 [1]  1  2  3  4  5  6  7  8  9 10
\end{CodeOutput}
\end{CodeChunk}



\end{document}

